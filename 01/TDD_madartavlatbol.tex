\section{TDD madártávlatból}

\begin{frame}{}
  \begin{quote}
    Debugging is twice as hard as writing the code in the first
    place. Therefore, if you write the code as cleverly as possible,
    you are, by definition, not smart enough to debug it.
  \end{quote}
  \begin{flushright}
    \textit{Brian Kernighan}
  \end{flushright}
\end{frame}

\subsection{Tesztvezérelt fejlesztés}
Kódolás $\Rightarrow$ Tesztelés
\pause

\begin{frame}{Problémák}
  \begin{itemize}[<+->]
  \item Az idő legalább fele szükséges a hibakereséshez
  \item Kiszámíthatatlan a hossza, tehát nem ütemezhető
  \item Egy hiba kijavítása más problémákat leplezhet le
  \item Sosem lehetünk igazán biztosak, hogy nincs már több hiba
  \item Nehéz a regresszió megállapítása
  \end{itemize}
\end{frame}

\begin{frame}{Megoldás}
  \begin{itemize}[<+->]
  \item Felismerték, hogy a rövidebb ciklusok kevesebb problémát eredményeznek
  \item Az automatizált tesztelés időt és pénzt takarít meg.
  \item A mellékhatások így gyorsabban felismerhetőek.
  \item A hosszú hibakeresés eliminálható
  \end{itemize}

  \pause
  A TDD egy hatékony módszer arra, hogy a tesztelést a fejlesztésbe szőjük.
\end{frame}

\subsection{Miért van szükségünk TDD-re}

\begin{frame}{A Zune bug}
  A zenelejátszó 2008. december 31-én egy napig üzemképtelen volt. Ez
  egy szökőév utolsó napja. A hibát egyetlen tesztesettel ki lehetett
  volna védeni.

  (zunebug.c)
\end{frame}

\begin{frame}{}
  Emberek vagyunk, követünk el hibákat, a programozás pedig rendkívül
  összetett tevékenység.

  \pause

  Egyéb indokok mellet a TDD egy remek módszer arra, hogy programjaink
  szervezett módon az elvárásoknak megfelelően működjenek, valamint
  hogy automatizált teszteseteink legyenek, amelyek hosszú távon
  ellenőrzik a működőképességet.
\end{frame}

\subsection{Mi az a tesztvezérelt fejlesztés}

\begin{frame}{}
  \begin{itemize}[<+->]
  \item \textbf{Nem} tesztelési módszer.
  \item Technikailag inkremetális fejlesztési mód.
  \item Nem írunk üzemi kódot addig, amíg nincs egy teszt, ami ellenőrzi annak működését.
  \item A kulcs az automatizálásban rejlik. A korábbi üzemi kód is tesztelt.
  \item Segít a fejlesztőknek megfelelő döntéseket hozni.
  \item A tesztek megmutatják az üzemi kód elvárt működését
  \item Segít a következő megoldandó feladatra koncentrálni
  \item Jó móka
  \end{itemize}
\end{frame}

\subsection{A TDD fizikája}

\begin{frame}{TDD $\Leftrightarrow$ DLP}
  \begin{itemize}[<+->]
  \item Mi a kész fogalma?
  \item DLP $\rightarrow$ Lassú visszacsatolás
  \item Nem világos gyökér ok
  \item Bizonyos működések a hibás kódra építenek
  \item Becsülhetetlen idő
  \end{itemize}
\end{frame}

\begin{frame}{Kép}
  Figure 1

  $T_d$: discover time \\
  $T_{find}$: find time \\
  $T_{fix}$: fix time \\
  $T_d \nearrow \Rightarrow T_{find} \nearrow$
\end{frame}

\subsection{A TDD mikrociklus}
\begin{frame}{Apró lépések}
  \begin{enumerate}[<+->]
  \item Adj hozzá egy kis teszt
  \item Futtasd a teszteteket, az új elbukik
  \item Módosítsd az üzemi kódot annyira, hogy a teszt ne bukjon el
  \item Futtasd a teszteket, az új teszt sem bukik el.
  \item Takarítás (refactoring)
  \end{enumerate}

  \pause
  $10 sec < \Sigma t < 5 min$
\end{frame}

\begin{frame}{Refactoring}
  A kód több kell, hogy legyen, mint pusztán a helyes működés alapja.

  \pause

  Tisztán kell tartani, hogy a későbbi változások könnyen elvégezhetőek legyenek.

  \pause

  Nem ``dolgokat'' készítünk, hanem dokumentációt, ami alapján
  ``dolgokat'' lehet készíteni. Szoftver esetében a programkód a
  teljes értékű dokumentáció.
\end{frame}

\begin{frame}{A ciklus}
  \textit{Red-Green-Refactor}
\end{frame}

\subsection{TDD előnyök}

\begin{frame}
  Időbe telik megtanulni.

  \pause

  \begin{itemize}[<+->]
  \item Kevesebb hiba
  \item Rövidebb hibakeresés
  \item Kevesebb mellékhatásból adódó hiba
  \item Dokumentáció, ami nem hazudik
  \item Benső béke
  \item Fejlett felépítés
  \item Folyamatjelző
  \item Móka és kacagás
  \end{itemize}

\end{frame}

\subsection{Előnyök beágyazott rendszerekhez}


%%% Local Variables:
%%% mode: latex
%%% TeX-master: "prog_eloadas_01"
%%% End:
