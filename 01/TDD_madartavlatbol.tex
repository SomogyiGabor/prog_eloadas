\section{TDD madártávlatból}

\begin{frame}{}
  \begin{quote}
    Debugging is twice as hard as writing the code in the first
    place. Therefore, if you write the code as cleverly as possible,
    you are, by definition, not smart enough to debug it.
  \end{quote}
\textit{Brian Kernighan}
\end{frame}

\subsection{Kódolás $\Rightarrow$ Tesztelés}

\begin{frame}{Problémák}
  \begin{itemize}[<+->]
  \item Az idő legalább fele szükséges a hibakereséshez
  \item Kiszámíthatatlan a hossza, tehát nem ütemezhető
  \item Egy hiba kijavítása más problémákat leplezhet le
  \item Sosem lehetünk igazán biztosak, hogy nincs már több hiba
  \item Nehéz a regresszió megállapítása
  \end{itemize}
\end{frame}

\begin{frame}{Megoldás}
  \begin{itemize}[<+->]
  \item Felismerték, hogy a rövidebb ciklusok kevesebb problémát eredményeznek
  \item Az automatizált tesztelés időt és pénzt takarít meg.
  \item A mellékhatások így gyorsabban felismerhetőek.
  \item A hosszú hibakeresés eliminálható
  \end{itemize}

  \pause
  A TDD egy hatékony módszer arra, hogy a tesztelést a fejlesztésbe szőjük.
\end{frame}
