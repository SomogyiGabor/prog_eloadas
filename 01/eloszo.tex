\section{Bevezetés}

\subsection{Mielőtt belevágnánk}
\begin{frame}{Mielőtt belevágnánk}
  Mint minden tanulási folyamathoz, a következőkre lesz szükségunk
  \begin{itemize}[<+->]
  \item Türelem
  \item Elszántság
  \item Bizalom
  \end{itemize}
\end{frame}

\begin{frame}{TDD}
  TDD = \textbf{T}est \textbf{D}riven \textbf{D}evelopment
\end{frame}

\subsection{Programfejlesztési stratégiák}
\begin{frame}{Programfejlesztési stratégiák}
  \only<1>{
    Írunk egyszerre sok kódot, madj kezdjünk el megbizonyosodni róla,
    hogy valóban működik-e?  \\
    Valaki valamikor csinált valamit, aminek már nem tudjuk, hogy mi
    az értelme, de ha hozzányúlunk, nem működik semmi.
  }

  \only<2>{
    Ha kiderül egy hiba (megbukik egy teszt), akkor az csak az utóbbi
    pár perc munkájának eredménye lehet.
  }
\end{frame}

\begin{frame}{A TDD előnyei és hátrányai}
  \begin{itemize}[<+->]
  \item Előnyök
    \begin{itemize}
    \item A hibák gyorsan kiderülnek
    \item Kontrollált a környezet amelyben egy modult tesztelünk
    \item Határfeltételek könnyen tesztelhetők
    \end{itemize}
  \item Hátrányok
    \begin{itemize}
    \item ``Lassú''
    \end{itemize}
  \end{itemize}
\end{frame}

\begin{frame}{Mi az a TDD?}
  A tesztek bizonyos értelmezés szerint a kód dokumentációját
  jelentik. Így lehet a legkönnyebben megmutatni a tesztelt kód
  klienseinek szempontjából, hogy az aktuálisan fejlesztett részt
  hogyan kell használni.
\end{frame}

\subsection{Mire számíthatunk?}

\begin{frame}{Mire számíthatunk?}
 A következő témákat fogjuk érinteni:

  \begin{itemize}[<+->]
    \item Rengeteg kód
    \item Tesztelés
    \item Tervezési elméletek
    \item Refaktorálás
    \item Kód szagok
    \item Munka örökölt kóddal
    \item Tervezési minták
    \item Tesztelési minták
    \item stb
  \end{itemize}

\end{frame}

\begin{frame}{}
  \begin{quotation}
    So, if you are a pragmatic embedded engineer who
    lives in the real world and codes close to the metal, then, yes, this
    book is for you. You've picked it up and read this far. Now finish
    what you started and read the rest of it.
  \end{quotation}
  \begin{flushright}
    \textit{Robert C. Martin}
  \end{flushright}
\end{frame}

\begin{frame}{Fejlesztés hardver nélkül}
  \only<1>{A tervezett kibocsadási időpont már meg van, de a szoftver még sehol,
    hiszen a hardver nincs kész, nem döntöttük el, hogy legyen-e OS, es ha
    igebm milyen.}

  \only<2>{A cél hadrver szűk keresztmetszet. Gyakran a HW és a SW egyszerre
    fejlődik és nem állnak egymás rendelkezésére. Ha ez nem lenne elég, a
    HW és a SW is tartalmaz hibákat, gyakran nehéz meghatározni, hogy
    melyik rész a felelős. Máskor a cél hardver annyira drága, hogy nem
    lehet mindenkinél külön példány belőle, így a fejlesztőknek várakozni
    kell, az idő pedig drága.}
\end{frame}

\subsection{A könyv tagolása}

\begin{frame}{A könyv tagolása}
  \begin{enumerate}[<+->]
  \item
    \begin{itemize}
    \item Rövid bevezetés a TDD-be
    \item Teszt keretrendszerek megismerése.
    \item Első modulunk tesztvezérelt fejlesztése.
    \item Felmerülő kérdések megválaszolása
    \end{itemize}

  \item
    \begin{itemize}
    \item Kódrészlet tesztelése, amely más kódrészleteket használ
    \item Hogyan lehet a tesztelendő kódot elszigetelni a függőségeitől?
    \item Teszt hasonmás, álca objektum bevezetése
    \end{itemize}

  \item
    \begin{itemize}
    \item Fontos tervezési elvek, amelyek segítenek jobb kódot írni
    \item Haladó C technikák, amelyekkel tesztelhető és rugalmas programokat készíthetünk
    \item Kódújratervezés
    \item Munka örökölt kóddal
    \end{itemize}
  \end{enumerate}
\end{frame}

%%% Local Variables:
%%% mode: latex
%%% TeX-master: "prog_eloadas_01"
%%% End:
